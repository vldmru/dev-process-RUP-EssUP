\documentclass[letterpaper]{report}
\usepackage[utf8]{inputenc}
\usepackage[T1,T2A]{fontenc}
\usepackage[utf8]{inputenc}
\usepackage[english,bulgarian,ukrainian,russian]{babel}

\title{Ітеративний процес розробки програмного забезпечення Rational Unified Process (RUP) та основні поняття Essential Unified Process (EssUP)}
\author{Владислав Марущак}
\date{25 вересня 2023р.}

\begin{document}
	\maketitle

	\tableofcontents

	\section*{Вступ}

		Розробка програмного забезпечення є складним та багатоетапним процесом, який вимагає систематичного підходу для досягнення успішних результатів. У цьому контексті виникає потреба в ефективних методологіях розробки, які допомагають забезпечити високу якість продукту, ефективність розробки та зменшити ризики. Однією з таких методологій є Rational Unified Process (RUP) та Essential Unified Process (EssUP). У цій доповіді ми розглянемо основні аспекти цих методологій, їх відмінності, переваги та основні поняття.

	\section{Rational Unified Process (RUP)}  

		\subsection{Огляд RUP}

			Rational Unified Process (RUP) - це ітеративний та інкрементальний процес розробки програмного забезпечення. Він був розроблений компанією Rational Software Corporation (придбано IBM) і став популярним методом для керування проектами розробки програмного забезпечення.

			Основними характеристиками RUP є:

		\begin{itemize}
			\item Ітеративність: Процес розробки розбито на ітерації, кожна з яких включає у себе певний набір фаз і завдань. Кожна ітерація вирішує певну частину функціональності продукту.
			\item Інкрементність: Продукт розробляється поетапно, додаванням нових функцій та функціональностей на кожній ітерації.
			\item Архітектурний підхід: RUP визначає важливість архітектури в розробці програмного забезпечення і надає настанови щодо її розвитку.
			\item Управління ризиками: Методологія RUP акцентує на розпізнанні та управлінні ризиками під час розробки.
			\item Клієнтський орієнтований підхід: Постійний зв'язок з клієнтом і активна участь замовника в процесі розробки.
		\end{itemize}

		\subsection{Фази RUP}

			RUP складається з кількох фаз, кожна з яких має свої завдання і вимоги. Основні фази RUP включають:

			\begin{enumerate}
				\item Фаза Завдань (Inception)
					У цій фазі визначаються основні цілі проекту, обговорюються ідеї та вимоги клієнта. Головною метою є визначення того, чи є проект життєздатним, і розроблення початкового плану проекту.
				\item Фаза Розробки (Elaboration)
					Ця фаза спрямована на розробку детальної архітектури продукту, визначення ключових компонентів та технологій. Розробляється докладний план реалізації проекту.
				\item Фаза Конструкції (Construction)
					У фазі конструкції здійснюється фактична реалізація продукту, написання коду та тестування. Функціональність додається ітеративно і інкрементально.
			\end{enumerate}

		\subsection{Основні принципи RUP}

			RUP ґрунтується на кількох ключових принципах:

			\begin{enumerate}
				\item Ітерації і інкременти
					Розробка програмного забезпечення поділена на короткі ітерації, під час яких додається нова функціональність. Це дозволяє замовнику швидко бачити прогрес і вносити зміни в специфікацію продукту.
				\item Архітектурна орієнтованість
					RUP визнає важливість архітектури та надає настанови щодо її розвитку на початкових етапах проекту.
				\item Управління ризиками
					Методологія RUP покладає великий акцент на розпізнанні і управлінні ризиками під час розробки, що дозволяє підтримувати проект в контрольованому стані.
				\item Зацікавлені сторони (Stakeholders)
					RUP активно залучає зацікавлених сторін у процес розробки, забезпечуючи постійний обмін інформацією з клієнтом, командою розробки і іншими учасниками проекту.
			\end{enumerate}

		\subsection{Переваги та недоліки RUP}

			\begin{enumerate}
				\item Переваги RUP
					\begin{itemize}
						\item Ітеративний та інкрементальний підхід сприяє ранньому виявленню помилок і змінам в вимогах.
						\item Акцент на архітектурній орієнтованості сприяє створенню стабільної та легко розширюваної системи.
						\item Залучення зацікавлених сторін сприяє зменшенню ризиків невідповідності продукту вимогам клієнта.
					\end{itemize}
				\item Недоліки RUP
					\begin{itemize}
						\item Високі вимоги до документації можуть призвести до перевитрат часу на папіркову роботу.
						\item Завдяки своїй складності, RUP не підходить для невеликих проектів або команд з обмеженими ресурсами.
						\item Вимагає досвідченого керівника проекту для ефективного впровадження.
					\end{itemize}
			\end{enumerate}

	\section{Essential Unified Process (EssUP)}

		\subsection{Огляд EssUP}

			Essential Unified Process (EssUP) - це ітеративний процес розробки програмного забезпечення, який був створений з метою спростити і розпространити управління проектами та розробкою ПЗ. EssUP побудований на базі RUP, але спрощений та адаптований для широкого кола проектів.

			Основні характеристики EssUP включають:

			\begin{itemize}
				\item Простота та скорочення зайвого бюрократичного апарату: EssUP намагається зменшити обсяг документації та формалізму, зберігаючи при цьому основні принципи RUP.
				\item Адаптованість: EssUP може бути адаптований до різних видів проектів, включаючи невеликі та середні.
				\item Спрощена архітектура процесу: EssUP зменшує кількість фаз та завдань, спрощуючи процес розробки.
			\end{itemize}

		\subsection{Фази EssUP}

			EssUP має дві основні фази:

			\begin{enumerate}
				\item  Фаза Ініціації (Inception)
					На початковому етапі фази ініціації визначаються основні цілі та обмеження проекту, а також складається перший план робіт.
				\item Фаза Виконання (Construction)
					У фазі виконання відбувається реалізація проекту. Ця фаза поділяється на ітерації, кожна з яких додає новий функціонал до продукту.
			\end{enumerate}

		\subsection{Основні принципи EssUP}

			EssUP зберігає основні принципи RUP, але додатково покладає акцент на наступних аспектах:

			\begin{enumerate}
				\item Простота
					EssUP спрощує процес розробки, зменшуючи кількість документації та формалізму.
				\item Адаптованість
					EssUP може бути адаптований до конкретних потреб проекту та команди розробників.
				\item Використання найкращих практик
					EEssUP пропагує використання найкращих практик розробки програмного забезпечення, таких як тестування, контроль версій, та інші.
			\end{enumerate}

		\subsection{Переваги та недоліки EssUP}

			\begin{enumerate}
				\item Переваги EssUP
					\begin{itemize}
						\item Спрощує процес розробки, що робить його більш доступним для менших команд та проектів.
						\item Забезпечує адаптованість до різних вимог та потреб.
						\item Використовує найкращі практики розробки програмного забезпечення.
					\end{itemize}
				\item Недоліки EssUP
					\begin{itemize}
						\item Може не підходити для складних проектів, де необхідна велика кількість документації і контролю.
						\item Може вимагати досвідченого керівника проекту для успішного впровадження.
					\end{itemize}
			\end{enumerate}

	\section{Порівняння RUP і EssUP}

		\subsection{Схожість RUP і EssUP}
			\begin{itemize}
				\item Обидва методи базуються на ітеративному підході до розробки програмного забезпечення.
				\item Обидва надають велику увагу архітектурному дизайну та управлінню ризиками.
				\item Використовує найкращі практики розробки програмного забезпечення.
			\end{itemize}

		\subsection{Відмінності RUP і EssUP}
			\begin{itemize}
				\item RUP є більш складним та більш вимогливим до документації методом, тоді як EssUP спрощує цей аспект.
				\item EssUP спрямований на менші та більш адаптовані проекти, тоді як RUP підходить для різних розмірів проектів, включаючи великі корпоративні.
				\item EssUP більше акцентується на простоту та використання найкращих практик.
			\end{itemize}

	\section{Висновки}

		У цій доповіді розгорнуто дві ітеративні методології розробки програмного забезпечення: Rational Unified Process (RUP) та Essential Unified Process (EssUP).
		RUP є більш складним та вимогливим до документації методом, який підходить для великих проектів з великою кількістю учасників. Він акцентує на архітектурному дизайні та управлінні ризиками.
		EssUP, натомість, спрощує процес розробки та надає можливість адаптувати методологію для різних проектів. Він покладає акцент на простоту та використання найкращих практик.
		Вибір між цими двома методологіями залежить від конкретних потреб та характеристик проекту. Керівництво проектом повинно враховувати розмір проекту, рівень складності, доступні ресурси та інші фактори для вибору найбільш підходящого методу.

\end{document}